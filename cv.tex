\PassOptionsToPackage{hyphens}{url}
\documentclass[11pt,a4paper,sans]{moderncv}

% ===== Stil / språk (pdfLaTeX) =====
\moderncvstyle{classic}
\moderncvcolor{blue}
\usepackage[scale=0.86]{geometry}
\usepackage[norsk]{babel}
\usepackage[T1]{fontenc}
\usepackage[utf8]{inputenc}
\usepackage{lmodern}

% Bedre linjebryting/lenker
\usepackage{microtype}
\usepackage[hidelinks]{hyperref}
\Urlmuskip=0mu plus 2mu
\emergencystretch=2em
\sloppy


% ===== Layout-tweaks =====
\setlength{\hintscolumnwidth}{2.8cm}         % venstrekolonnen

% ===== Personlig info =====
\name{Christian}{[Etternavn]}
\title{Student -- Fremtidig Dataingeniør}
\address{Halden, Norge}{}
\email{din@epost.no}
\phone[mobile]{+47 [telefon]}
% Brukernavn i stedet for fulle URL-er for best linjebryting:
\social[github]{dittbrukernavn}
\social[linkedin]{dittprofil}
% Valgfritt bilde: legg filen ved eller behold kommentert
% \photo[70pt][0.4pt]{cvphoto.jpg}

\begin{document}
\makecvtitle

% ===== Sammendrag =====
\section{Introduksjon / Kort sammendrag}
\cvitem{}{
Student og \textbf{fremtidig dataingeniør} med spesialisering i \textbf{backend-utvikling} og systemarkitektur.
Løsningsorientert og klar for utfordringer, med fokus på å bygge effektive, skalerbare og robuste kjernesystemer i \textbf{.NET / C\#}.
Søker jobb eller praksisplass hvor jeg kan bidra med teknisk dybde og initiativ.
}

% ===== Teknisk kompetanse =====
\section{Teknisk kompetanse (Backend \& Verktøy)}
\cvitem{Kjernestakk}{Ekspertise i \textbf{.NET / C\#} med vekt på \textbf{skalerbarhet} og \textbf{robusthet}.}
\cvitem{Data}{Erfaring med relasjonsdatabaser og \textbf{SQL}-optimalisering. Håndtering av ustrukturert data (\textbf{JSON}).}
\cvitem{DevOps}{\textbf{Git}, \textbf{Azure DevOps} (CI/CD), profesjonell kodeflyt i \textbf{Visual Studio}.}
\cvitem{Frontend}{Grunnleggende \textbf{HTML/CSS} og \textbf{Blazor} for integrasjon mot backend.}

% ===== Prosjekter =====
\section{Utvalgte prosjekter (Praktisk erfaring)}

\cventry{2024--2025}{\textbf{OsmToolkit -- OSM Data Processing Library}}{Selvstendig prosjekt}{}{}{
\begin{itemize}
  \item \textbf{Beskrivelse:} \textbf{.NET/C\#-bibliotek} for effektiv håndtering av OpenStreetMap (OSM)-data.
  \item \textbf{Rolle:} Systemarkitekt og datamodelleringsansvarlig. Implementerte serialisering/deserialisering (JSON/XML) og skrev enhetstester.
  \item \textbf{Resultat:} Robust bibliotek med fokus på ytelsesoptimalisering og pipeline-integrasjon.
\end{itemize}
}

\cventry{2023--nå}{\textbf{Sim Racing Stats Dashboard -- Sim Racing Standings}}{Personlig prosjekt}{}{}{
\begin{itemize}
  \item \textbf{Beskrivelse:} \textbf{.NET}-løsning som henter, behandler og aggregerer data fra iRacing-API.
  \item \textbf{Mål:} Generere liga-statistikk og presentere data i et \textbf{dashboard} for intuitiv visualisering.
  \item \textbf{Utfordringer:} Høy dataflythastighet og begrenset API-dokumentasjon.
\end{itemize}
}

% ===== Nøkkelferdigheter =====
\section{Nøkkelferdigheter \& Egenskaper}
\cvitem{Løsningsorientert}{Trives med komplekse, nye problemstillinger som krever nøyaktighet og innovasjon.}
\cvitem{Mestring}{Kontinuerlig læring og eksperimentering for målrettet fullføring.}
\cvitem{Strategisk}{Analytisk og planmessig; erfaring fra sim racing og gaming overføres til utviklingsarbeid.}

% ===== Utdanning =====
\section{Utdanning}
\cventry{2023--2026 (pågående)}{Bachelor i Dataingeniør}{Høgskolen i Østfold, Halden}{}{}{
Relevante emner: Algoritmer, Databaser, Systemutvikling, Nettverk, Statistikk, Programvarearkitektur.
}

% ===== Språk =====
\section{Språk}
\cvitem{Norsk}{Morsmål}
\cvitem{Engelsk}{Flytende (skriftlig og muntlig)}

% ===== Interesser =====
\section{Interesser}
\cvitem{}{Sim Racing \quad\textbullet\quad Programmering \quad\textbullet\quad Gaming \quad\textbullet\quad OSM/Kartdata \quad\textbullet\quad Backend-design \quad\textbullet\quad C\#/.NET}

% ===== Ekstra info / lenker =====
\extrainfo{
\href{https://github.com/dittbrukernavn}{GitHub: dittbrukernavn} \quad\textbullet\quad
\href{https://www.linkedin.com/in/dittprofil}{LinkedIn: dittprofil}
}

\end{document}
